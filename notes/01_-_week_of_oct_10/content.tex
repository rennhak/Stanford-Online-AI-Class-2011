\section{Unit 1}

\subsection{Purpose of this Class}
  \begin{itemize}
    \item to teach you the very basics of artificial intelligence, so you can talk and understand
    the basic tools of the trade
    \item to excite you of the field
  \end{itemize}

\subsection{Structure}
  \begin{enumerate}
    \item Videos
    \item Quizzes
    \item Answer videos for the quizzes and exams before
    \item Assignments or Exams\footnote{Please see ai-class.org}
  \end{enumerate}

\subsection{Mini Quiz}
  An AI program is called...\\

  \begin{itemize}
    \renewcommand{\labelitemi}{$\square$}
    \item Wetware
    \item Formula
    \item Intelligent Agent
  \end{itemize}

  \answer{The last item is the right one}

\subsection{Intelligent Agent}

  We have an intelligent Agent, which interacts with the environment as seen in Figure
  \ref{fig:Intelligent_Agent}. The Agent can perceive the state of the environment through its
  sensors and it can effect its state through its actuators.\\

  \begin{figure}[!h]
      \centering
      \fbox{
      \includegraphics[width=8cm]{fig1/fig1.eps.gz}
      }
      \caption{Intelligent Agent: Perception Action Cycle}
      \label{fig:Intelligent_Agent}
  \end{figure}

  The big question is the function which maps sensors to actuators. That is the control policy for
  the agent. This whole class deals with how the agent makes decisions which it can carry out with
  its actuators based on past sensor data.

  These decisions take place many times in the loop of the environment feedback, this is called the
  {\bf Perception Action Cycle}.\\

\subsection{Mini Quiz}

  AI has successfully been used in (choose one or more)\\

  \begin{itemize}
    \renewcommand{\labelitemi}{$\square$}
    \item Finance
    \item Robotics
    \item Games
    \item Medicine
    \item The Web
    \item None of them
  \end{itemize}

  \answer{All except the last}


\subsection{AI in Finance}

  \begin{figure}[!h]
      \centering
      \fbox{
      \includegraphics[width=8cm]{fig2/fig2.eps.gz}
      }
      \caption{AI in Finance}
      \label{fig:AI_in_Finance}
  \end{figure}



% vim:ts=2:tw=100:wm=100
